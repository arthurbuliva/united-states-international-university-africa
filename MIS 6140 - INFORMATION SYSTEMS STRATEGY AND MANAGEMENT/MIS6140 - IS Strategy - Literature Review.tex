\documentclass[12pt]{article}
%\usepackage{fullpage}
\usepackage{multicol}
%\usepackage{times}
\usepackage{palatino}

%\usepackage{draftwatermark}
%\SetWatermarkText{\textbf{CONFIDENTIAL}}
%\SetWatermarkScale{3}

\setlength{\parindent}{0em}
\setlength{\parskip}{1em}

\begin{document}
\thispagestyle{empty}
\vspace*{\fill}
\noindent
\begin{center}
\textbf{
{\huge Information \& Communications Systems Strategy: A Look at the Information Strategy of The United Nations in Kenya}
}\\
\vspace{0.2in}
Arthur Buliva\\
\emph{United States International University - Africa}\\
abuliva@usiu.ac.ke
\end{center}
\vspace*{\fill}

\newpage
\clearpage
\pagenumbering{arabic}

\section*{Abstract}

Non-governmental organizations (NGOs) are a key feature in many developing nations, among which Kenya falls. This paper will examine how the United nations as an example of an NGO - through its sub agencies like the World Food Programme (WFP), United Nations Children Fund (UNICEF) among others - uses Information Systems in order to carry our its charter within Kenya.

At present there are over 60 offices of UN Funds, Programmes and Agencies in or operating from Kenya \cite{unon}. Total number of UN staff working in Kenya is close to 4600. In addition some common services (e.g. Kenya Security Coordination) are provided to Bretton Woods Institutions (IMF, World Bank) and associated international organizations e.g. IOM, ICC etc \ldots 

UN assistance to Kenya is realized on the basis of the United Nations Development Assistance Framework (UNDAF) \cite{undaf} which emphasizes the principles of UN Delivering as One (DaO), aimed at ensuring Government ownership, demonstrated through UNDAF’s full alignment to Government priorities and planning cycles, as well as internal coherence among UN agencies and programmes operating in Kenya.

The UNDAF narrative includes five recommended sections: Introduction and Country Context, UNDAF Results, Resource Estimates, Implementation Arrangements, and Monitoring and Evaluation as well as a Results and Resources Annex \footnote{Inclusion of a section on Initiatives outside UNDAF was considered but it was UNCT’s agreement that all initiatives would fall within the utilized framework \cite{undaf}.}.

On 16 December 1963, Kenya became a member of the United Nations \cite{uninkenya}. With this membership, Kenya and the UN had to work together in co-operation. Indeed, the United Nations had to come up with a strategy in order to build up its presence in the country. Part of that strategy needed to be a communications strategy that makes use of Information Technology in order to communicate its cause. 

The United Nations has largely embraced IT strategy to support its mandate. Just as business conditions keep changing rapidly, the UN also needs to adopt an effectively IT strategy to suit its needs and also be in line with its vision and mission statement.

The research was done using literature found in published United nations journals, official communiqu\'es academic journals and online search engines. 

Even though the UN has various semi-autonomous agencies, they all eventually align under One UN to provide governance and guidance. It is necessary that UN in Kenya take all the necessary measures to align its IT strategy with the global UN strategy.

After a review of target audiences by individual UN agencies, the following categories have been identified as most important to the majority of the agencies: \cite{undaf}

\begin{itemize}
\item Government and its ministries/agencies
\item Parliament
\item Mass media
\item Civil society organizations
\item Donor community
\item Academia, research institutions
\item Youth organizations and networks
\item Community based organizations
\item Private sector (an emerging partner)
\end{itemize}

\section*{Introduction}
Non-governmental organizations (NGOs) generally drive the development of developing economies especially in their rural areas. As in many sectors of the modern economy, the NGOs are turning to IT for solutions for their systems to run effectively. Within the UN, this is also applicable. Most manual operations have been eliminated and many operations now handled by a kind of IT system or another. 

The United Nations, however, has embraced innovations at slower pace compared to other organizations. This is because results of innovation take time and the UN has had to consider regulation and risk mitigation issues before adopting any new technology. However, the pace is picking up and structured financial products like derivatives are the result of product innovation.

The demographics and diversity of the UN also makes the embracing of Information Strategies a slow process especially because of the reliance on outsourcing and a need to control an increasingly complex environment. The aging workforce also experiences a big need on training in order for them to remain relevant, competent and efficient. Effort is required to enable the different entities to manage these changes.

\subsection*{IT Strategy: Then, Now and The Future}
\textit{Information Systems (IS) strategy} is a comprehensive plan that information technology management professionals use to guide their organizations. Like any other strategy, this strategy is all about a road plan:

\begin{enumerate}
\item Where are we now
\item Where do we want to go?
\item How will we get there?
\end{enumerate}

In the past, IT managers had inadequate knowledge of the business strategies and business managers also has very little understanding of the potential IT had. Most IT plans were technical and reactive rather than proactive and strategic. 

In the course of the last decade, a changing world has forced the United nations to \emph{Do More With Less} \cite{undaf}. This has made IT to be the go-to strategy in order to do more with less. Having much of the travel costs reduced, for example, has made video conferencing the defacto meeting method for many conferences.

Because the UN charter was developed before the Information Age really picked up, many of the processes were designed to be manual with faxes flying all over the place. However, Information and Communication Technology has enabled the automation of many processes, controls, and information production using computers, telecommunications and software.

Recently, the United Nations Secretariat introduced an enterprise resource planning system called Umoja \cite{umoja}. This technology has enabled self-service facilities that eliminates much manual paper-pushing.

This development has enabled the UN to provide a much more efficient and secure services that can be accessed wherever one may be, through the use of the Internet. 

The UN holds extremely sensitive information and depend almost fully on IT as their core technology.  This information has to be strategically kept in order that any unauthorized access is unthinkable.

\subsection*{Alignment of IT with the UN mandate}
When IT serves the main business need, then IT makes sense and is said to be aligned with the business. There is a need for appropriate and timely application of IT to the overall business objectives. 

Some of the business objectives of the United nations include development programmes, refugee support, peacekeeping missions, internal HR, finance and procurement. Sometimes, a compromize needs to be made in order for such an alignment to occur.


The UN office in Nairobi has relied on outsourcing because the expertise it needed was not available within the internal workforce. However, much of the workforce has learned the necessary skills and indeed newer blood has been injected from the market. This has made outsourcing to be a secondary workforce and not primary. The addition of skills such as internal applications development and maintenance has enabled IT to understand the business more, thereby helping IT align with business.

Aligning IT and business strategies is a complex process that requires all stakeholders' input. In the UN in particular, it involves the various sub agencies of the UN all aligning to fulfilling the mandate of the entire UN. The organization structure and accountability and strategy formation processes revolve intra-agency and inter-agency participation worldwide.

Based on the nature of the work of the UN, there is a very strong need of good communication in addition to the good set of IT and business skills plus good communication. These approaches must be made more robust and be useful in the larger firm.

In order to make sure that the communication strategy was viable, the members of the UN made a plan to make use of a set of 'building blocks' that constituted the following theme areas:

\begin{itemize}
\item External Communication: Newsletters, Advocacy, Media (including social Media) Events and Visibility,
\item Internal Communication: UN intranet, UN Induction Package and Recognition and Awards Programme.
\item Emergency Communication: Guiding principles and standard operating procedures (SOPs)
\end{itemize}


\section*{Success Factors in Developing an IT Strategy}

In order to develop an efficient and effective IT strategy, there are a few factors that need to be considered by the UN in Kenya:

\subsection*{Revisiting The UN Business Model}
The UN does not work in isolation. It collaborates closely with Government and civil society partners. Consequently, the communications strategy emphasizes on the need for cultivating existing partnerships and collaborations as well as encouraging formation of new partnerships.

These collaborative relationships and strategic partnerships aim to help the UN to maximize their record in addressing various development and human rights related issues. The relationships also provide a great opportunity to promote a participatory approach to planning, development, implementation and monitoring of the UN’s technical assistance programmes, thus making them more sustainable. Strategic partnerships will also help minimise duplication of efforts.

In most cases, business models and strategies are confused with each other. The business model describes how different parts of the business fit. Initially, all employees need to have their focus on the value the UN is intent in creating. Only then will it be easier to develop strategies on how to deliver value.

Indeed, it is imperative that both IT employees and business employees understand in detail how the United nations as a whole works. In many aspects, most employees see IT as a necessary evil instead of realising how it is an enabler of business. Consequently, there has been misalignment and conflicting strategies of IT and business in the past. Despite the willingness of business managers to understand the impacts of IT in business, the IT team must also make an effort and translate their ideas in a business language and avoid using IT jargon in their presentations. (Smith et al.,2007)

\subsection*{Adoption of Strategic Themes}
For a long time, IT strategy used to work for personal projects. By extension, this also worked for individual separate UN entities. With time, however, large projects have come up within the UN that separate entities have to join hands.

Other issues affect everyone in a similar way, such that it only makes sense that joint effort is made in developing centralized IT systems. A good example is a security alert system that uses SMS based messages to relay a variety of information. Such a system indeed needs to have a common approach within all parties involved.

\subsection*{Support from Senior Management}
The bureaucratic nature of the UN makes it such that progress is only made when decisions are made on a top-down approach. Therefore, it means that the executive management are in the fore-front of IT decision making. It is thus necessary for the managers to meet from time to time with other senior business managers to discuss both IT and business strategies.

It is important that the people with the necessary expertise to be involved in order to  develop the IT strategy. This may includes the business managers, heads of the various agencies and other key stakeholders.

\subsection*{Balancing IT and Business Costs}
There are many different ways that the UN can make use of technology in order to improve business wile creating opportunities. Even though this is good, it may pose a challenge when it comes to developing an effective IT strategy and the resources and skills needed to manage the opportunities are not also readily available.

There is a need to properly allocate budget for IT. Many a times, IT budget has been a matter of left-over budget from other departments. Consequently, there is a need for analyzing and finding the most cost-effective way of providing IT services to their organizations. Part of this method can be outsourcing of non-critical IT services such as printing, IT support or Internet services.

\subsection*{Focusing on the UN Charter}
Every department within the United Nations plays a role in delivering the UN mandate. However, there may be disagreement on how this business value should be measured. The IT team should come up with an IT agenda that will achieve both business and IT related metrics in order to derive business value. 

According to Balwant, (2013) the IT team should also readjust their R\&D from pure technology to a catalyst of innovation, so that business can align to it.  There should be an allocation of IT Research and development from the budget instead of being considered an IT expenditure



\subsection*{Challenges}
Both IT and UN agency heads are making efforts in order to ensure that both IT and business strategies are aligned. However there are still barriers that inhibit development of strategies. There is lack of a supportive governance structure; There are no formal structures for managing inter-dependencies among the different UN agencies; There is no clarity on who is responsible for turning IT strategies into IT plans.

Clearly business and IT are not moving at the same pace strategically.


\subsection*{Conclusion}
One of the big successes of the communication strategies of the United Nations system in Kenya is the presence of the headquarters of a major agency here - the United Nations Environment Programme (UNEP). This has ensured that various other agencies have a "big brother"  who does the strategies for them to replicate. 

Further to that, the other smaller UN offices rely not only on the infrastructure of UNEP, but also the framework that it has built regarding external and internal communication channels.

Because of the very nature of the UN, some functions overlap among the various agencies and this may cause confusion especially if the strategies also overlap. An example is that the United nations Development Programme, UNDP, strategizes on using the government mechanisms in order to foster development, whereas United Nations World Food Programme involves itself with food distribution. UNDP is teaching how to fish while at the same time WFP is giving fish.

The 2007/2008 post-election violence gave a good opportunity to showcase how the communication strategies of the UN can be used for the good of the population. Coming together with the international community showed how the strategy of a collaboration is what is needed in order to move as one.

Much of the Kenyan communication infrastructure such as the Internet, newspaper distribution channels, etc. have concentrated their efforts around the urban periphery. Much of rural Kenya usually lags behind in getting information. Consequently, the strategic information needed to get the rural areas up-to-speed is delivered late or never. Therefore through alternative means should the strategy be focusing on getting the message across.

A study of the UN has revealed the need to improve internal communication, documentation and information sharing through a joint online platform. The present UN Communication Strategy therefore proposes to establish a joint web-based format in the form of the UN Intranet which could facilitate informing all the UN staff in the country about the UN vision, help streamline the document exchange on common matters, promote the culture of sharing substantive programme information, better coordinate the UN’s advocacy and  outreach activities and increase staff awareness on UN activities.

The UN Resident Coordinator’s Office will be responsible for regular maintenance of the content of the UN Intranet with inputs from the UNCG members or designated Communication Focal Points of the UN agencies. As for the UN calendar of events, UNIC and UN communication focal points will be responsible for regularly updating planned events on the Intranet concerning their respective agencies including training, project site visits and major missions.

\vspace{1cm}
\hrule width1.5in height1pt
\vspace{1cm}

\newpage

\begin{thebibliography}{9}

\bibitem{census}
Kenya National Bureau of Statistics
\emph{The 2009 Kenya Population and Housing Census}
Published August, 2010

\bibitem{unon}
United Nations Office at Nairobi
\emph{United Nations Office at Nairobi}
http://www.unon.org
Accessed \today

\bibitem{uzbekistan}
United Nations
\emph{United Nations Communication Strategy}
Published by the United Nations
http://www.un.org
Accessed \today

\bibitem{indonesia}
Yanuar Nugroho
\emph{NGOs, the Internet and sustainable rural development: The case of Indonesia}
Journal: Information, Communication and Society
Volume 13, Issue 1, February 2010
Published online: 10 Feb 2010

\bibitem{undaf}
United Nations Development Programme
\emph{United Nations Development Assistance Framework 2014-2018}
Office at the UN Resident Coordinator, Kenya

\bibitem{griffiths}
John Ward and Pat Griffiths
\emph{Information Systems evolution; Strategic Planning for Information Systems}
1997

\bibitem{uninkenya}
The UN in Kenya
\emph{The Permanent Mission of The Republic of Kenya to the United Nations}
Published by the Mission of The Republic of Kenya to the United Nations
https://www.un.int/kenya/kenya/kenya\\\%E2\%80\%99s-voice-united-nations-50-years
Accessed \today

\bibitem{remenyi}
DSJ Remenyi 
\emph{Strategic Information systems; SISP}
1992

\bibitem{earl}
Earl
\emph{ISS and IM strategy; Information Management}
1998

\bibitem{wendy}
Wendy Robson
\emph{ISS frameworks; Strategic Management of IS}
1998
  
\bibitem{chen}
Daniel Q. Chen et al.
\emph{Information Systems Strategy: Conceptualization, Measurement and Implications}



\end{thebibliography}

\newpage
\thispagestyle{empty}
\vspace*{\fill}
THIS PAGE HAS BEEN INTENTIONALLY LEFT BLANK
\vspace*{\fill}

\end{document}

%\begin{comment}
\bibitem{shamim}
Shamim Ziaee and Xavier N. Fernando, Senior member of IEEE,
\emph{Broadband over Power Line: An Overview},
Ryerson University, Toronto, ON.

\bibitem{seema}
Seema M. Sing, Esq
\emph{Broadband Over Power Lines}
 State of New Jersey
%\end{comment}
