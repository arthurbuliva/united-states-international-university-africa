\documentclass[11pt]{article}
\usepackage[table]{xcolor}
\usepackage{booktabs}
\usepackage{lipsum}
\usepackage{fullpage}
\usepackage{multicol}
%\usepackage{times}
%\usepackage{palatino}
\begin{document}

\noindent
\begin{center}
\textbf{
{\huge Power Line Communication: The Challenges}
}\\
\vspace{0.2in}
Arthur Buliva\\
\emph{United States International University - Africa}\\
abuliva@usiu.ac.ke
\end{center}



\begin{multicols}{2}

\section*{The challenges of Power Line Communications}
Powerlines are a particularly difficult communications media. Noise levels are excessive and attenuation at the frequencies that are of interest are unusually large. The resistance of the material and the noise levels vary quite unpredictably. 

There are two main issues associated with PLC, \emph{cabling configuration} and \emph{interference}. For powerline to be useful, any devices being connected ought to be on the same circuit physically. This may not always the case, because many houses have separate circuits  in ordere reduce the load on each cable. There may be a separate circuit for the sockets, a separate one for the lights and a third for high power appliances such as cookers and boilers. In these circumstances, PLC may present challenges in linking the entire home.

The second problem is that of interference. Interference may be atmospheric, caused by lightning or static charges, or that from other devices within the circuit. Computers and computing devices such as printers are the worst offenders here. Motors from appliances also cause a regular waveform interference pattern. With an increase in the number of powerline devices in a particular area, together with powerline technology being used by multiple households; for example in a flat (apartment complex); the signals increasingly risk interference and degradation. Since the data frequency is being shared, there is a general reduction in data throughput..

One big controversy that revolves around powerline communications relates to the interference that it causes for other electrical equipment, since powerline signals radiate into the air so have a tendency to interfere with other electrical devices like short wave and medium wave radio receivers. In addition to that, newer models of powerline communication devices have also been shown to interfere with FM radio receivers too.

One major area where wireless technology consistently wins out over powerline is in incorporating into the home network mobile devices like the Apple iPad and other tablet devices which have no ethernet connection. As video is also increasingly being used by these devices which have no fixed location, wireless is undoubtedly the best option.

\begin{thebibliography}{9}

\bibitem{etm}
The Institution of Engineering and Technology,
\emph{Powerline versus WiFi - the pros and cons},
Engineering and Technology magazine
http://eandt.theiet.org/contribute/consumer-technology/powerline-vs-wifi.cfm

\bibitem{cyclopaedia}
cyclopaedia.net
\emph{Power Line Communication}
http://www.iem.uni-due.de/~vinck/reference-papers/PLC-encyclopy-ferrera-atal.pdf

\end{thebibliography}


\end{multicols}

\end{document}
